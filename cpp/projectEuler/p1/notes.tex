\documentclass[11pt, oneside]{article}   	% use "amsart" instead of "article" for AMSLaTeX format
\usepackage{geometry}                		% See geometry.pdf to learn the layout options. There are lots.
\geometry{letterpaper}                   		% ... or a4paper or a5paper or ... 
%\geometry{landscape}                		% Activate for for rotated page geometry
%\usepackage[parfill]{parskip}    		% Activate to begin paragraphs with an empty line rather than an indent
\usepackage{graphicx}				% Use pdf, png, jpg, or eps§ with pdflatex; use eps in DVI mode
								% TeX will automatically convert eps --> pdf in pdflatex		
\usepackage{amssymb}

\title{Project Euler: Problem 1}
\author{Adrian Lange}
%\date{}							% Activate to display a given date or no date

\begin{document}
\maketitle
%\section{}
%\subsection{}

\section{Multiples of 3 and 5}
If we list all the natural numbers below 10 that are multiples of 3 or 5, we get 3, 5, 6 and 9. The sum of these multiples is 23. 
Find the sum of all the multiples of 3 or 5 below 1000.

\section{Solution}
Well, the brute force algorithm is easy. Just a for loop up to 1000 summing up if the modulus of 3 or 5 is 0. The answer is 233168.

But, Wikipedia has nicely provided me with a much more clever solution. We can easily figure out how many integers below 1000 (max 999) are divisible by 3 or 5, 999/3 = 333 and 999/5 = 199. Then, it's easy to get the sums of these numbers by using the formula,
\begin{equation}
	1 + 2 + 3 + 4 + 5 + ... = \sum_{i=1}^{n} i = \frac{n(n+1)}{2} \;.
\end{equation}
Then, we've got
\begin{equation}
	{\rm sum} = 3 * \frac{999(999+1)}{2} + 5 * \frac{199(199+1)}{2} \;
\end{equation}
But, that double counts the numbers that are divisible by both 3 and 5 ({\em i.e.} divisible by 15. So, we just correct for that
\begin{equation}
	{\rm sum} = 3 * \frac{999(999+1)}{2} + 5 * \frac{199(199+1)}{2} - 15 * \frac{66(66+1)}{2} = 233168 \;.
\end{equation}

\end{document}  